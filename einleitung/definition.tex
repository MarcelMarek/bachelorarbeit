\chapter{Einleitung} \label{sec:definition}
\section{MongoDB}
MongoDB ist eine hoch skalierbare und flexible Dokumentendatenbank mit effizienter Abfrage und Indizierung. MongoDB speichert Daten in flexiblen, JSON-ähnlichen Dokumenten, das heißt Felder können von Dokument zu Dokument variieren und die Datenstruktur kann im Laufe der Zeit geändert werden.

\section{ExpressJS}
Express ist ein minimales und flexibles NodeJS-Webanwendungs-Framework, das robuste Funktionen für Web- und mobile Anwendungen bietet. Es stellt eine Vielzahl von HTTP-Utility-Methoden und Middleware zur Verfügung, mit denen sich schnell und einfach eine robuste API erstellen lässt. Express bietet eine vielfältige Schicht grundlegender Webanwendungsfunktionen.

\section{VueJS}
Laut der Dokumentation ist Vue.js ein progressives JavaScript-Framework für die Erstellung von Benutzeroberflächen. Es ist zugänglich, leistungsfähig und vielseitig bei der Erstellung von Single-Page-Webanwendungen.
VueJS konzentriert sich auf die Ansichtsschicht. Es hat eine sehr einfache Lernkurve mit einer einfachen API, was es zu einem der beliebtesten Frameworks macht.

\section{NodeJS}
Node.js ist eine Open-Source-Laufzeitumgebung und -Bibliothek, die zur Ausführung von Webanwendungen außerhalb des Browsers des Kunden verwendet wird. Sie wird hauptsächlich für die serverseitige Programmierung verwendet. Es ist asynchron, ereignisgesteuert und hoch skalierbar, um Server und Datenbanken zu schreiben.

