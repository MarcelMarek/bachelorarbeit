\chapter{Design and Implementation}
-> Datensätze vorstellen
-> Tensorflow Project überprüfen

math.confusion:matrix .> erkennen falscher bilder

neue  Gliederung -> Mit Koch absprechen / Neuer Termin mit Koch
-> Neuer Termin mit Koch 
-> Schauen, dass das Thema noch nicht "erforscht" wurde
-> Hat es Vorteile Daten zu verbessern?
-> Wie verhalten sich Daten, wenn man spiegelt / manipuliert
-> Wie verbessert sich die learning rate?
-> Die große herausforderung sind sinnvolle Trainingsdaten zu bekommen - Wir bauen ein Toll und schauen was rauskommt.
-> Forschungsfragen auf zwei reduzieren
1. Tool Plattform 2. Funktionen Augmentation (Spiegeln, automatisch Suchen)Beschreiben was wir gedacht haben, wieso zum Vervollständigen gewählt. Welches dieser Verfahren ist das beste. Wie bekomme ich saubere Datenlage? 

Forschungsfragen können allgemeiner sein
-> Welche Ansätze gibt es
-> Nicht fest Datensatz festlegen (Klassenunterscheidung statt "Hund" und "Katze") -> Wieviel Prozent sind notwendig, um zu unterscheiden zu können
-> Mehr Fokus auf Augmentation statt Labeling Plattform
-> Alle kritischen Stellen rauswerfen

Forschungsfrage: Vergleich unterschiedlicher MEthoden zur Augmentation
Forschungsfrage: Erst bei Interpretation statistischer Teil. Ergebnis neutral.  

tensorflow gute bücher
Gibt es eventuell schon viel zu viele papers -> nicht genau meins, sondern andere richtung. Ist das gut oder schlecht?

Data Scraping gibt es schon eine BA? -> Warum BA

Zielsetzung der Arbeit gut? Vorlesen

Zuviele Forschungsfragen? -> Soll nur "Was gibt es als Lösungsansätze" als Forschungsfrage, wollte professioneller klingen.

Hintegrund und Motivation -> Durch den Anwendungsfall innerhalb der Firma pep.digital werden die Mitarbeiter beim Labeln helfen (daher Webplattform gewählt)

-> ggbfs als Quelle auch BA -> Quelle in BA nachschauen

-> Zügig klarheit geschaffen und koch befragen (scope hat sich geändert)
