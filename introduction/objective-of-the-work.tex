\section{Zielsetzung der Arbeit}\label{sec:introduction-objectives}
Das Ziel ist es einen ''Computer Vision''-Datensatz zu erweitern, bis sich die Genauigkeit eines KI-Modells erhöht. Das Modell ist auf die Unterscheidung zwischen den beiden Gruppen ''Katze'' oder ''Hund'' trainiert.  Der Datensatzes teilt die Daten ebenfalls in ''Hund'' und ''Katze'' ein . Der Umgang mit Fehlern und Lücken soll untersucht werden und verschiedene Lösungsansätze zur Steigerung der Trefferquote angeboten werden. 

Dieses Projekt erarbeitet zur Überprüfung der erarbeiteten Lösungsansätzen eine Webplattform zur Bildkennzeichnung und -manipulation. Eine Mischung zwischen automatischen Lösungen und händischen Tools für die beschriftenden Mitarbeiter werden bereitgestellt. Inwiefern das KI-Modell nach der Bearbeitung durch die zur Verfügung gestellten Tools auf der Webseite höhere Trefferquoten erzielt, soll zweiter Untersuchungsgegenstand der Bachelorarbeit sein.

 Es resultieren drei Forschungsfragen, die zu untersuchen sind:

   Was sind die Ursachen für unvollständige Datensätze und welche Herausforderungen in Bezug auf Datenqualität und -integrität entstehen bei der Verarbeitung unvollständiger Datensätze?
   Ab welcher Anzahl von fehlenden Datenpunkten wird die Ergänzung mit zusätzlichen Daten empfohlen und was sind effektive Strategien zur Lösung von Datenlücken, wie etwa zusätzlicher Datenerfassung oder die Nutzung von externen Datenquellen? 
   Welche Werkzeuge sind verfügbar, um durch Datenbereinigung und -erweiterung die Trefferzahl zwischen 'Das ist ein Hund' oder 'Das ist eine Katze' zu erhöhen und wie kann die praktische Implementierung dieser Lösungsansätze erfolgen?
   Welche Auswirkungen haben Modifikationen eines Datensatzes auf die Lernrate und die Genauigkeit der Klassifizierung?
   ''Inwieweit kann Datenaugmentation die Qualität der Trainingsdaten verbessern?''

